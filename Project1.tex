% LaTeX Article Template - customizing page format
%
% LaTeX document uses 10-point fonts by default.  To use
% 11-point or 12-point fonts, use \documentclass[11pt]{article}
% or \documentclass[12pt]{article}.
\documentclass{article}

% Set left margin - The default is 1 inch, so the following 
% command sets a 1.25-inch left margin.
\setlength{\oddsidemargin}{0.25in}

% Set width of the text - What is left will be the right margin.
% In this case, right margin is 8.5in - 1.25in - 6in = 1.25in.
\setlength{\textwidth}{6in}

% Set top margin - The default is 1 inch, so the following 
% command sets a 0.75-inch top margin.
\setlength{\topmargin}{-0.25in}

% Set height of the text - What is left will be the bottom margin.
% In this case, bottom margin is 11in - 0.75in - 9.5in = 0.75in
\setlength{\textheight}{8in}
\usepackage{fancyhdr}
\usepackage{float}
\usepackage{mathtools}
\usepackage{amsmath}
\usepackage{amssymb}
\setlength{\parskip}{5pt} 
\pagestyle{fancyplain}
% Set the beginning of a LaTeX document
\begin{document}

\lhead{Drew Remmenga MATH 408}
\rhead{Project \#1}
%\lhead{Independent Study}
%\rhead{R Lab}

\begin{enumerate}

\item \begin{enumerate}
\item 
$
T(x)=e^{(5/2)}(1+\frac{(x-5)}{2}+\frac{(x-5)^2}{8}+\frac{(x-5)^3}{48}+\frac{(x-5)^4}{384}) \\
$


$ L(x)=e^{1/2}\frac{(x-3)(x-5)(x-7)(x-9)}{384}+ e^{3/2}\frac{(x-1)(x-5)(x-7)(x-9)}{-96}+e^{5/2}\frac{(x-1)(x-3)(x-7)(x-9)}{64}+$\\$ e^{7/2}\frac{(x-1)(x-3)(x-5)(x-9)}{-96}+ e^{9/2}\frac{(x-1)(x-3)(x-5)(x-7)}{384}$

\item 
\item Taylor Error $=\left| f^{n+1}(\xi) \right|_{max} \frac{(x-c)^{(n+1)}}{(n+1)!}$ \\
Taylor Error $=\left| e^{\xi/2} \right|_{max} \frac{(x-5)^5}{3840}$ \\
Lagrange Error $\displaystyle = \left| f^{n+1}(\xi)\right |_{max} \frac{1}{(n+1)!} \prod_{i=0}^{n} (x-x_{i})$\\
Lagrange Error$=\left| e^{\xi/2} \right|_{max} \frac{(x-1)(x-3)(x-5)(x-7)(x-9)}{3840}$ 
\item 
\item
\item Our Error estimates are seceral dozen times larger than the actual error indicated. This is because our first truncated term is much larger than the next term if we went one term further. This is due to the factorial part in particular when calculating our taylor and lagrange errors. 
\end{enumerate}
\item 
\begin{equation*}
\begin{split}
\displaystyle L_{k}(x) & =\frac{1}{a_{k}} \prod_{j=1, j\neq k}^{n+1} (x-x_{j}) \\
ln(L_{k}(x)) & =-ln(a_{k})+ \sum_{j=1, j \neq k}^{n+1}ln(x-x_{j}) \\
\frac{L'_{k}(x)}{L_{k}(x)} &= \sum_{j=1, j \neq k}^{n+1} \frac{1}{(x-x_{j})} \\
L'_{k}(x) &= L_{k}(x)\sum_{j=1, j \neq k}^{n+1} \frac{1}{(x-x_{j})}
\end{split}
\end{equation*}
\item Given:
\begin{equation*}
\begin{split}
\displaystyle u(h) =\mathcal{O}(v(h))  \text{ as } h \to 0 \\
\implies \\
\exists C \in \mathbb R_{>0} : \left| u(h) \right| < C\left| v(h) \right| \\
\end{split}
\end{equation*}
We find without loss of generality that:
\begin{equation*}
\begin{split}
\alpha u(h) & = \mathcal{O}(v(h)) \\
\end{split}
\end{equation*}
Since we are dealing solely in magnitudes $\alpha$ can be negative and we still receive:
\begin{equation*}
\begin{split}
\left| \alpha \right| \left| u(h) \right| &< C\left| v(h) \right| \\
\alpha \left| u(h) \right| &< C\left| v(h) \right| \\
 \left| u(h) \right| &< \frac{C}{\alpha} \left| v(h) \right| \\
\end{split}
\end{equation*}
For:
\begin{equation*}
\begin{split}
\frac{C}{\alpha} \in \mathbb R_{>0} \\
\end{split}
\end{equation*}
\item
\begin{equation*}
\begin{split}
p(x) &= \frac{(x-x_{2})(x-x_{3})u(x_{1})}{(x_{1}-x_{2})(x_{1}-x_{3})} +  \frac{(x-x_{1})(x-x_{3})u(x_{2})}{(x_{2}-x_{1})(x_{2}-x_{3})}+ \frac{(x-x_{1})(x-x_{2})u(x_{3})}{(x_{3}-x_{1})(x_{3}-x_{2})}\\
p'(x) &= \frac{(2x-(x_{2}+x_{3}))u(x_{1})}{(x_{1}-x_{2})(x_{1}-x_{3})}+\frac{(2x-(x_{1}+x_{3}))u(x_{2})}{(x_{2}-x_{1})(x_{2}-x_{3})}+\frac{(2x-(x_{1}+x_{2}))u(x_{3})}{(x_{3}-x_{1})(x_{3}-x_{1})} \\
p''(x) & = \frac{2u(x_{1})}{(x_{1}-x_{2})(x_{1}-x_{3})}+\frac{2u(x_{2})}{(x_{2}-x_{1})(x_{2}-x_{3})}+\frac{2u(x_{3})}{(x_{3}-x_{1})(x_{3}-x_{2})}\\
\end{split}
\end{equation*}
Set:
\begin{equation*}
\begin{split}
x_{1} = \bar{x}-h, x_{2} = \bar{x}, x_{3} = \bar{x}+h
\end{split}
\end{equation*}
So:
\begin{equation*}
\begin{split}
p''(\bar{x}) & = \frac{2u(\bar{x}-h)}{(\bar{x}-h-\bar{x})(\bar{x}-h-\bar{x}-h)}+\frac{2u(\bar{x})}{(\bar{x}-\bar{x}+h)(\bar{x}-\bar{x}-h)}+\frac{2u(\bar{x}+h)}{(\bar{x}+h-\bar{x}+h)(\bar{x}+h-\bar{x})}\\
p''(\bar{x}) & = \frac{2u(\bar{x}-h)}{2h^{2}}+\frac{2u(\bar{x})}{-h^{2}}+\frac{2u(\bar{x}+h)}{2h^{2}}\\
\end{split}
\end{equation*}
\item
\begin{enumerate}
\item fdcoeffF.m running with xbar = 1, second derivative: \\
 h = 1 returns [-0.0833,1.3333,-2.5,1.3333,-0.0833] \\
h = 2 returns [-0.1042, 0.3048, -0.4583, 0.2667, -0.0089] 
\item
\item
\end{enumerate}
\item
\begin{enumerate}
\item 
\begin{equation*}
\begin{split}
y_{1}(t) &= u(t) \\
y_{2}(t) & = u'(t) \\
y_{3}(t) &= v(t) \\
y_{4}(t) &= v'(t) \\
y'_{1}(t) = u'(t) &= y_{2}(t) \\
y'_{2}(t) = u''(t) &= -\frac{y_{1}(t)}{(y_{1}(t)^{2}+y_{3}(t)^{2})^{3/2}}  \\
y'_{3}(t) = v'(t) &= y_{4}(t) \\
y'_{4}(t) = v''(t) &= -\frac{y_{3}(t)}{(y_{1}(t)^{2}+y_{3}(t)^{2})^{3/2}}  \\
\end{split}
\end{equation*}
\item
\item
\item

\end{enumerate}
\end{enumerate}



\end{document}
